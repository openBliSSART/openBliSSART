%
% vim:set expandtab tabstop=4 shiftwidth=4:
%


\section{Data Organization}

openBliSSART's data storage consists of a SQLite database \cite{SQLite} (in the
{\tt db} directory of the installation tree) in conjunction with an archive of
binary files (in the {\tt storage} directory). The database stores information
about the available objects (such as components generated by the NMF), their
features and class labels, while the object data itself is externalized to
binary files.

Generally\footnote{The separation process can also be run in a ``volatile'' mode
  that does not store anything. This is useful for example if the result of a
  NMF separation should be output as WAV files. See section
  \ref{section:septool} for details.}, when processing audio files, e.g. by FFT
and/or NMF, openBliSSART saves information about the separation process, such as
the name of the input file, the number of components, the STFT parameters
etc. in a respective \emph{process} entity. Furthermore, the computed objects
(such as NMF components) are saved as \emph{classification objects}. Each
classification object consists of one or more \emph{data descriptors} which
describe data like spectral vectors or phase matrices.

\paragraph{Classification Objects} openBliSSART currently creates and handles
the following types of classification objects:
\begin{center}
  \begin{tabular}{|p{.3\textwidth}|p{.6\textwidth}|}
    \hline
    NMD component & generated by applying STFT and NMD or NMF
    to an audio file \\
    Spectrogram & generated by applying STFT to an audio file \\
    \hline
  \end{tabular}
\end{center}

\paragraph{Data Descriptors} The following types of data descriptors are used:
\begin{center}
  \begin{tabular}{|p{.3\textwidth}|p{.6\textwidth}|}
    \hline
    Magnitude matrix & the magnitude spectrogram of an audio file \\
    Phase matrix & the phase spectrogram of an audio file \\
    Spectrum & a magnitude spectrum, generated by NMF or NMD from an audio file; a vector in case of NMF, or a matrix in case of NMD \\
    Gains & a gains vector, generated by NMF or NMD from an audio file \\
    \hline
  \end{tabular}
\end{center}
Note that it is perfectly valid for a data descriptor to occur in relation to
more than one classification object. For example, each classification object
generated by a NMF process contains a reference to the phase matrix of the
original signal so as to be able to re-synthesize wave files from one or more
components. The phase matrix, however, is stored only once.

Each data descriptor is associated with a separation process with a unique ID.
These IDs can for instance be found out by looking at the process listing in
the browser application, and are needed for component feature extraction
as well as data export.

\paragraph{Features, Responses and Labels} Data descriptors relate to
\emph{features} which are used during classification. A \emph{response} assigns
classification objects to \emph{labels}. Classification is done using features
from the data descriptors that make up the classification objects in the
response.\\

\noindent The browser (\ref{section:browser}) can be used to conveniently
explore the database structure.



\section{Source separation algorithms}

TODO: IDP, DA